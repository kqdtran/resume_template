% My LaTeX resume using res.cls
\documentclass[margin]{res}
\usepackage{enumitem}
\usepackage{url}
\usepackage[pdfborder={0 0 0}]{hyperref}
\renewcommand{\familydefault}{\sfdefault}
\setlength{\topmargin}{-0.60in}
\setlength{\oddsidemargin}{-0.68in}
\setlength{\parskip}{3mm plus1mm minus1mm}
\usepackage{setspace}
\singlespacing

\begin{document}
% Center the name over the entire width of resume:
 \moveleft.5\hoffset\centerline{\large\bf Khoa Q.D. Tran}
% Draw a horizontal line the whole width of resume:
 \moveleft\hoffset\vbox{\hrule width\resumewidth height 0.5pt}\smallskip
 \moveleft.5\hoffset\centerline{2508 Ridge Rd. Apt 3, Berkeley, CA 94709}
 \moveleft.5\hoffset\centerline{Email: \href{mailto:khoatran@berkeley.edu}{\tt{khoatran@berkeley.edu}} - Phone: (831)402-3491}
 \moveleft.5\hoffset\centerline{\href{http://kqdtran.github.io/}{\tt{kqdtran.github.io}} - \href{https://github.com/kqdtran}{\tt{github.com/kqdtran}}}
 
\begin{resume}
\section{EDUCATION} 
\textbf{University of California, Berkeley} \hfill Fall 2012 - Present \\
Bachelor of Arts, Computer Science. Cumulative UC GPA: 3.73 \\
Expected graduation date: May 2014 

\textbf{Monterey Peninsula College} \hfill Fall 2010 - Spring 2012 \\
Associate of Arts, Computer Science and Mathematics. Cumulative GPA: 4.00
 
\section{RELEVANT \\ COURSEWORK} 
\begin{tabular}{@{}l l l}
Data Structures & Discrete Mathematics & Probability \& Risk Analysis for Engineers \\ 
Machine Structures & Algorithms & Industrial \& Commercial Database Systems \\ 
Artificial Intelligence \textit{(Fall 13)} & Applied NLP \textit{(Fall 13)} & Computer Networking \textit{(Fall 13)} \\
\end{tabular}
 
\section{EXPERIENCE} 
\textbf{Computer Science Intern} \hfill June 2013 - Present \\
\textit{Ocean Tomo, LLC} \smallskip
\begin{itemize}[leftmargin=10pt]
%\itemsep -1pt %reduce space between items
\item Improve workflow efficiency by implementing features that allow employees to handle ``conflict checks" and job requests online, using Play Framework 2, Bootstrap, jQuery, MySQL, Circumflex ORM, and Elasticsearch
\item Create interactive visualizations and reports with D3.js using data extracted from an Access database
\end{itemize}

\textbf{Content Developer} \hfill March 2013 - Present \\
\textit{\href{https://mathapedia.com/}{Mathapedia Research Group}, EECS Department, UC Berkeley} \smallskip
\begin{itemize}[leftmargin=10pt]
\item Build interactive contents for \textit{CS70: Discrete Mathematics \& Probability Theory} and \textit{EE149: Intro to Embedded Systems} using \LaTeX{}, Javascript, and MathJax
\end{itemize}

\textbf{Reader/Grader} \hfill February 2013 - Present\\
\textit{EECS Department, UC Berkeley} \smallskip
\begin{itemize}[leftmargin=10pt]
\item Grade weekly problem sets for 400+ students in \textit{CS70: Discrete Mathematics \& Probability Theory}
\item Collaborate with TAs and other Readers to assist students in weekly office hour and on online discussion forum 
\end{itemize}

\textbf{Calculus Tutor} \hfill June 2012 - August 2012 \\
\textit{Math Learning Center, Monterey Peninsula College} \smallskip
\begin{itemize}[leftmargin=10pt]
\item Collaborated with instructors to provide one-on-one and in-group tutoring to 30+ students taking Calculus
\end{itemize}

\section{PROJECTS} 
\textbf{Twitter Sentiment Analysis} \hfill Python, Twitter API v1.1, OAuth2
\begin{itemize}[leftmargin=10pt]
\item Opinion-mining system that determines the polarities of real-time tweets, and from there answers questions like the happiest US state, or the most popular hashtags
\end{itemize}

\textbf{Project: Juice Database} \hfill MS Access, SQL
\begin{itemize}[leftmargin=10pt]
\item Relational database management system with built-in SQL queries and \href{http://kqdtran.github.io/course/ieor115/slide.html}{web-based presentation slides} built for \textit{Project: Juice}, a San Francisco-based startup, in a team of 8 students
\end{itemize}

\textbf{Ninja Scraper} \hfill Python, Beautiful Soup
\begin{itemize}[leftmargin=10pt]
\item Command-line web scraper that automatically downloaded past exams for a selected course at UC Berkeley 
\end{itemize}

\textbf{Plagis} \hfill Java
\begin{itemize}[leftmargin=10pt]
\item Plagiarism detector that checked for similarities among homework submissions in an Introductory Programming class taught in Java
\end{itemize}

\section{TECHNICAL SKILLS}
\textbf{Languages}
\begin{itemize}[leftmargin=10pt]
\item \textit{Most experienced with} Scala, Java, and Python
\item \textit{Familiar with} C/C++, HTML, CSS, JavaScript, SQL, R, \LaTeX{}, and Bash Scripting
\end{itemize}

\textbf{Software}
\begin{itemize}[leftmargin=10pt]
\item \textit{Operating Systems}: Unix/Linux, Mac OS X, Windows 7/XP
\item \textit{Databases}: SQLite, MySQL, PostgreSQL, Microsoft Access
\item \textit{Frameworks \& Libraries}: Play Framework 2, Flask, Twitter Bootstrap, jQuery, D3.js
\item \textit{Other Tools}: Git, Heroku, Visual Studio, Eclipse, IntelliJ, Emacs
\end{itemize}

\end{resume}
\end{document}
